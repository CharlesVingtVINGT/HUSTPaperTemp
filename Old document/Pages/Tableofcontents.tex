%重新设置目录
\usepackage{tocloft}
\usepackage{hyperref}
\hypersetup{hidelinks} %去掉目录红框
\renewcommand{\cfttoctitlefont}{\hfill \heiti \zihao{-2} \bfseries}
\renewcommand{\contentsname}{目\hspace{2em}录}
\renewcommand{\cftaftertoctitle}{\hfill}
\renewcommand{\cftbeforeloftitleskip}{0.5em}
\renewcommand{\cftafterloftitleskip}{0.5em}
\renewcommand{\cftsecdotsep}{\cftdotsep} %设置Section引导点
\renewcommand{\cftbeforesecskip}{0em} %设置段间距
\renewcommand{\cftpartfont}{\songti \bfseries}%设置Part字体
\renewcommand{\cftsecfont}{\songti \bfseries}%设置Section字体

%使用条件语句
\usepackage{xstring}

%定义两个新语句
%第一个语句更改fancyhdr样式
\newcommand{\plainfooterstyle}[1]{
	\IfStrEqCase{#1}{{nopagenum}{}
		{pagenum}{\footstyle}
		{pagenumtoc}{\footstyle}}
}
%第二个语句在目录中添加目录标签
\newcommand{\addtoctotoc}[1]{
	\IfStrEqCase{#1}{{nopagenum}{}
		{pagenum}{}
		{pagenumtoc}{\phantomsection
			\addcontentsline{toc}{section}{目录}}}
}

%设置新的生成目录命令
\let \oldtableofcontens \tableofcontents
\newcommand{\maketoc}[1][nopagenum]{
	%修改hdr原plain格式
	\fancypagestyle{plain}{%
		\fancyhf{} %清空原有样式
		\headstyle
		\plainfooterstyle{#1}
	}
	%设置目录hdr和页码选项
	\clearpage
	\pagestyle{plain}
	\addtoctotoc{#1}
	\tableofcontents
	\clearpage
	\pagestyle{main}
}