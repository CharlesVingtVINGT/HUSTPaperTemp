\documentclass[UTF8]{ctexart}

\title{\bfseries \zihao{2} 华中科技大学\\本科生毕业论文模板\\使用说明}
\author{Skinaze}
\date{\today}

\usepackage{xltxtra}
\usepackage{doc}
\usepackage{enumitem}
%\setlist{noitemsep}
\RequirePackage{hyperref}
\hypersetup{colorlinks=true} 

\begin{document}
	\maketitle
	\tableofcontents
	\section{快速开始}
	本模板的基本使用方法如下:
	
	\begin{enumerate}[noitemsep]
		\item 下载并安装最新版本的MiK\TeX (推荐)或者\TeX\space Live\\
		\item 打开\TeX works,并设置默认编译工具(Edit->Preference->Typesetting->Processing Tools->Default)为\XeLaTeX+MakeIndex+\BibTeX\\
		\item 新建一个tex文档并保存在一个文件夹下,将Template文件夹下所有文件拷贝到该文件夹下\\
		\item 在新的tex文档中使用\verb|\documentclass{HustGraduPaper}|使用模板定义的样式
	\end{enumerate}
	
	\section{页面与章节}
	本模板针对论文内容设定了如下几个新的/重写的页面:
	
	\subsection{标题页}
	标题页面包括校名、论文题目和其他个人信息。
	
	使用方法:首先,需要在引言中插入进行个人信息设置如下:
	\begin{verbatim}
		\title{论文题目} %论文题目
		\author{作者姓名} %作者姓名
		\date{\today} %日期,默认当日
		\school{院系名称} %院系名称
		\classnum{专业班级} %专业班级
		\stunum {U201300000} %学号
		\instructor{指导教师姓名} %指导教师姓名
	\end{verbatim}
	之后在正文中使用命令\verb|\maketitle|生成标题页面。
	
	\subsection{声明与授权页}
	该页面包含学术声明内容和授权使用选项,本业主要内容无需手动更改。
	
	使用方法:在正文中使用命令
	\begin{verbatim}
		verb|\makestatement[保密年数]{empty/true/false}|
	\end{verbatim}
	生成声明与授权页面。该命令包含一个可选参数和一个必填参数,用于设置勾选保密或不保密。其中:empty为不勾选;true为保密,如选此项请填写保密年数;false为不保密。
	
	\subsection{中英文摘要环境}
	本模板设置了两个摘要环境,分别是针对中英文摘要。
	
	使用方法:对于中文摘要环境,这样使用
	\begin{verbatim}
		\begin{cnabstract}{关键词1;关键词2;关键词3}
		  这里是摘要内容
		\end{cnabstract}
	\end{verbatim}
	请注意中文关键词使用{\bfseries 中文分号}进行分割。对于英文摘要环境,这样使用
	\begin{verbatim}
		\begin{enabstract}{Key1; Key2; Key3}
		  Here is the content of the abstract.
		\end{enabstract}
	\end{verbatim}
	请注意英文关键词使用{\bfseries 英文分号+一个空格}进行分割。
	
	\subsection{目录}\label{subsec:toc}
	根据官方样式,目录包含中英文摘要、正文、致谢、参考文献和附录。本模板已经将上述内容加入到目录中,无需手动设置。
	
	使用方法:在正文中使用命令
	\begin{verbatim}
		\maketoc[nopagenum/pagenum/pagenumtoc]
	\end{verbatim}
	添加目录内容,其中:nopagenum指目录没有页码(默认值);pagenum指目录有页码;pagenumtoc指目录有页码,且目录两字出现在目录中。请注意本模板不会自动设置页码格式,具体页码设置方法,请参考“{\itshape ~\nameref{subsec:pagenum}}”。
	
	\subsection{正文}
	正文和标题的使用方法与\LaTeX 基本使用方法一致,在此不做过多说明。
	
	\subsection{致谢环境}
	本文设置了一个致谢环境,方便添加致谢并将其添加到目录中。使用方法:
	\begin{verbatim}
		\begin{thankpage}
		  这里是感谢的话。
		\end{thankpage}
	\end{verbatim}
	
	\subsection{参考文献}
	本模板参考文献采用\BibTeX 生成,因此需要制作独立的Bib文件。“*.bib”是一种数据库文件,其中包含了参考文献的基本信息,如此即可直接通过编译生成参考文件文字。
	
	文献数据库可以手工逐条录入,也可以从互联网上直接下载现成的文献数据库,很多电子期刊数据库网站会提供相应的\BibTeX 数据库文件或者\BibTeX 条目导出,Google\textsuperscript{\textregistered}  Scholar也免费提供此服务。\cite[\S 3.3]{9787121202087}
	
	如果希望手工录入,推荐安装JabRef管理参考文献,该系统可以直接搜索论文的DOI编号、书目的ISBN编号,甚至论文的名称,从而直接获得其相关信息,极大地方便了参考文献的录入。
	
	若要在本模板中使用参考文献,请这样使用:首先在文章末尾参考文献的地方使用
	\begin{verbatim}
		\bibliography{bib file name}
	\end{verbatim}
	Bib文件名中不需要加扩展名。之后可以通过命令
	\begin{verbatim}
		\cite{bib id}
	\end{verbatim}
	引用一个参考文献,被引用的参考文献会自动出现在文章末尾的参考文献中。也可以使用命令
	\begin{verbatim}
		\nocite{bib id}
	\end{verbatim}
	隐式引用一个参考文献,这样引用不会在文中标出参考文献号。当然如果你使用的是自己录入的Bib文件,你也可以通过命令
	\begin{verbatim}
		\nocite{*}
	\end{verbatim}
	直接将所有Bib文件中的参考文献列出。
	
	\subsection{附录}
	本附录使用了Appendix宏包,附录使用首先需要开启附录环境,之后可以用Section、Subsection、Subsubsection来构建附录的具体内容。附录的三级标签会在目录中显示,且一级标签会添加上附录二字。理论上可以在文章任意位置加入附录,但是推荐将所有附录放在文末。
	
	具体使用方式如下:
	\begin{verbatim}
		\begin{appendices}
		  \section{这是附录的第一级}
		  \subsection{这是附录的第二级}
		  \subsubsection{这是附录的第三级}
		\end{appendices}
	\end{verbatim}
	
	\section{其他常用环境}
	\subsection{图片环境}
	本模板保留了原来的图片环境figure的同时,添加了一个generalfig环境,方便添加居中的,带有标题的图片。其使用方法如下:
	\begin{verbatim}
		\begin{generalfig}[htbp]{图片标题}{fig:figlabel}
		   \includegraphics{mypic.png}
		 \end{generalfig}
	\end{verbatim}
	其中该环境的第二个参数是图片的位置,选择此处({\bfseries h} ere)、页顶({\bfseries t} op)、页底({\bfseries b} ottom)或者独立一页({\bfseries p} age)显示\textsuperscript{\cite[\S 5.3]{9787121202087}},默认选项是“htbp”;第三个参数是图片的标题;第四个参数是引用名称,你可以轻松使用
	\begin{verbatim}
		\autoref{fig:figlabel}
	\end{verbatim}
	引用该图片的编号,输出的效果是:“图2-1”。\verb|\includegraphics|命令用于引用一张图片,当然你也可以不使用该命令而通过\LaTeX 绘图宏包,如tikz,自行绘制图片。
	
	\subsection{表格环境}
	本模板保留了原来的表格环境table的同时,添加了一个generaltab环境,方便添加居中的带有标题的表格。其使用方法如下:
	\begin{verbatim}
		\begin{generaltab}{表格标题}{tab:tablabel}
		  \begin{tabularx}{\textwidth}{lCCC}
		    \toprule
		    序号&年龄&身高&体重\\
		    \midrule
		    1&14&156&42\\
		    2&16&158&45\\
		    3&14&162&48\\
		    4&15&163&50\\
		    \cmidrule{2-4}
		    平均&15&159.75&46.25\\
		    \bottomrule
		  \end{tabularx}
		\end{generaltab}
	\end{verbatim}
	其中该环境的第二个参数是表格的位置,选择此处({\bfseries h} ere)、页顶({\bfseries t} op)、页底({\bfseries b} ottom)或者独立一页({\bfseries p} age)显示\cite[\S 5.3]{9787121202087},默认选项是“htbp”;第三个参数是表格的标题;第四个参数是引用名称,你可以轻松使用
	\begin{verbatim}
	\autoref{tab:tablabel}
	\end{verbatim}
	引用该表格的编号,输出的效果是:“表2-1”。
	
	本模板还包含了tablularx库用于生成表格,并且定义了新的可变长度左中右(LCR)格式,具体tablularx的使用请参照\href{http://mirror.lzu.edu.cn/CTAN/macros/latex/required/tools/tabularx.pdf}{tabularx的文档}。另外,建议在论文中针对数据使用标准的三线表样式,就像上面使用的那样。
	
	\subsection{公式}
	本文保留了原来的有标签的公式环境equation,同时重新设置了其标签并使之带有章节号。公式的使用与正常\LaTeX 中的使用方法相同,在此不再赘述。
	
	\subsection{带编号列表和不带编号列表}
	本文保留了原来的带有编号的列表enumberate和不带编号的列表itemize,并取消了列表项之间的间距,具体使用方法和正常\LaTeX 中的使用方法相同,在此不再赘述。
	
	\section{其他注意事项}
	\subsection{页码问题}\label{subsec:pagenum}
	本模板没有自动设置页码的功能,因此需要使用者自己设置页码的样式和页码的开始。根据学校官方模板,摘要页采用大写罗马数字作为页码,因此在摘要环境开始前使用如下命令:
	\begin{verbatim}
		\clearpage %完成上一页,进入新的一页
		\pagenumbering{Roman} %摘要页码为大写罗马数字
	\end{verbatim}
	正文环境使用阿拉伯数字作为页码,因此在目录之后,第一节开始之前使用如下命令:
	\begin{verbatim}
		\clearpage %完成上一页,进入新的一页
		\pagenumbering{arabic} %正文页码为阿拉伯数字
	\end{verbatim}
	官方模板未对目录页码做以说明,本模板默认设置目录没有页码,也没有页脚,如需更改该选项,请参考本文中的~\nameref{subsec:toc}章节。如果上面说的不够直观,您也可以参考下文的~\nameref{sec:example}。
	
	\subsection{字体问题}
	本模板采用\XeLaTeX 进行编译,所以字体设置遵循xeCJK的默认设置,具体如下:
	\begin{description}
		\item[Mac OS X] 华文字库
		\item[Windows(Vista 及以后)]中易字库+ 微软雅黑
		\item[Windows (XP 及以前)] 中易字库
		\item[其他] Fandol字库\footnote{由马起园、苏杰、黄晨成等人开发的开源中文字体,参见:\url{https://www.ctan.org/pkg/fandol}。}
	\end{description}\cite[\S 4.3]{ctexdoc}
	考虑到学校官方模板并未对宋体、黑体的具体字体做以要求,本模板采用xeCJK的默认字体。
	
	但问题在于,学校官方模板标题页中会使用到华文中宋(STZhongsong)字体,该字体是微软\textsuperscript{\textregistered}公司出品的Office产品中的字体\footnote{该字体实为中国常州华文印刷新技术有限公司\textsuperscript{\texttrademark}开发。},为了避免版权纠纷,本模板未包含该字体。鉴于本人未在互联网上找到该字体的正版购买渠道,因此{\bfseries 如果需要使用本模板请安装Office软件。因下载安装使用盗版字体造成的版权纠纷与本人无关。}
	
	\section{一个小例子}\label{sec:example}
	为了方便使用,在这里提供一个简单的使用范例,范例中只有少量注释,请参考前文查看。本模板还包含了一个相对详细的例子,在“Example”文件夹下,如果需要也可参考该样例。
	\begin{verbatim}
		\documentclass{HustGraduPaper}
		
		\title{论文题目} %论文题目
		\author{作者姓名} %作者姓名
		\date{\today} %日期,默认当日
		\school{院系名称} %院系名称
		\classnum{专业班级} %专业班级
		\stunum {U201300000} %学号
		\instructor{指导教师姓名} %指导教师姓名
		
		\begin{document}
		  \maketitle %生成标题页
		  \makestatement{empty} %生成声明页
		  
		  \clearpage %结束上一页
		  \pagenumbering{Roman} %摘要页码为大写罗马数字
		  
		  \begin{cnabstract}{关键词1;关键词2;关键词3}
		    这里是摘要内容
		  \end{cnabstract}
		  \begin{enabstract}{Key1; Key2; Key3}
		    Here is the content of the abstract.
		  \end{enabstract}
		  
		  \maketoc %生成目录
		  
		  \clearpage %结束上一页
		  \pagenumbering{arabic} %正文页码为阿拉伯数字
		  
		  \section{第一节 The first Section}
		  \subsection{第一小节}
		  \subsubsection{第一小小节}
		  正文内容\cite{bibid} %这样引用参考文献
		  
		  \begin{thankpage}
		    感谢页面内容
		  \end{thankpage}
		  
		  \bibliography{Bibs/mybib} %生成参考文献
		  
		  \begin{appendices}
		    \section{这是第一个附录}
		    这里是附录环境,其中的section、subsection、subsubsection已经变为附录的样式,并且会以这种样式加入目录中
		    \subsection{附录可以有小节}
		    \subsubsection{附录中也可以有小小节}
		  \end{appendices}
		\end{document}
	\end{verbatim}
	
	\section{写在最后}
	\LaTeX 是一个强大的排版工具,本文所述内容只包含了本模板实现的主要功能,其他诸多功能不能尽述,希望使用者,尤其是初学者,能够针对性参考其他文档进行排版。
	
	本文如有描述不周的地方欢迎通过\href{mailto:me@stringblog.com}{邮件方式}联系我。或者访问本人的个人博客\url{https://stringblog.com/}。
	
	\phantomsection
	\addcontentsline{toc}{section}{参考文献}
	\bibliographystyle{plain}
	\bibliography{mybib}
	
\end{document}